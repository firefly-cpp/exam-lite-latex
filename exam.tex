% for appropriate date format, change babel language in .cls
\documentclass[slovene, exam]{iexam} %1: slovene, english; 2: exam, midterm

\begin{document}

% change according to your data
%-----------------------------------------------------------
% name of the subject
\subject{Portali in sistemi znanja}
% academic year
\academicyear{2022-2023}
% study program
\programme{ITK}
% id of exam
\examid{1}
% location of exam
\location{A-306}
% date of exam
\newdate{examdate}{18}{11}{2022}
% acronym of subject
\subacronym{PISZ}

%-----------------------------------------------------------

% create first page
%-----------------------------------------------------------
\makefirstpage
%-----------------------------------------------------------

% header of question pages
%-----------------------------------------------------------
\newpage % put question in a new page
\lhead{\displaydate{examdate}}
\rhead{\pacademicyear}
\chead{{\large\psubject}}
% space between head and first question (text)
\setlength{\headsep}{0.4in}
\cfoot{\thepage}
\setlength{\footskip}{30pt}
%-----------------------------------------------------------

% questions of the exam
% sytnax
% \questionn{What do you want to ask?}{No. of points}{conjugation}{vspace in cm}
%-----------------------------------------------------------


\question{1}
\questionn{Naštejte nekaj slabosti standardizacije podatkov.}{3}{e}{0}


\question{2}
\questionn{Zapišite glavne značilnosti inteligentnih sistemov.}{2}{i}{1}

\question{3}
\questionn{Naštejte vsaj 5 tehnik priprave podatkov in zapišite njihove značilnosti in za kakšen namen se določena tehnika uporablja?}{6}{}{0}

\begin{enumerate}[i]
    \item \alinea{Zakaj je pomembno izvesti pripravo podatkov preden začnemo učiti klasifikator?}{2}{i}{1}
    \item \alinea{Ali je vedno potrebno izvesti postopek priprave podatkov?}{2}{i}{1}
    \item \alinea{Kaj pomeni termin \textbf{podatkovna množica} in kako je sestavljena?}{2}{i} {1}
    \item \alinea{Ali je vedno potrebno izvesti postopek priprave podatkov?Ali je vedno potrebno izvesti postopek priprave podatkov?Ali je vedno potrebno izvesti postopek priprave podatkov?Ali je vedno potrebno izvesti postopek priprave podatkov?Ali je vedno potrebno izvesti postopek priprave podatkov? }{2}{i} {1}
\end{enumerate}

\questionn{Obkroži črko pred pravilnim odgovorom}{1}{}{0}
\begin{enumerate}
    \item Testni
    \item Pravilen
    \item Nepravilen
\end{enumerate}

\end{document}
