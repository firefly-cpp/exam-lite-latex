\documentclass{iexam}
\usepackage[slovene]{babel}
\usepackage[T1]{fontenc}
\usepackage[utf8]{inputenc}
\usepackage{framed}
\usepackage{lipsum}
\usepackage[iso,american]{isodate}

\begin{document}

\large

%%%%%%%%%%%%%%%%%%%%%%%%%%%%%%%%%%%%%%
%change according to your data

% name of the subject
\def \subject {Portali in sistemi znanja}
% academic year
\def \academicyear {2022-2023}
% date of exam
\def \examdate {15.11.2022}
% study program
\def \programme{ITK}
% location of exam
\def \location{A-306}
% id of exam
\def \examid{1}
% acronym of subject
\def \subacronym{PISZ}

% prepare header
\aboutexam{Predmet: \subject \\
			   Študijsko leto: \academicyear \\
			   Študijski program: \programme \\
			   Izpitni rok: \examid \\
			   Prostor: \location \\
			   Datum: \examdate}

%%%%%%%%%%%%%%%%%%%%%%%%%%%%%%%%%%%%%%

\noindent\makebox[\linewidth]{\rule{0.9\paperwidth}{0.4pt}}

\begin{center}IZPIT\end{center}

\vspace*{2\baselineskip}

\noindent\underline{NAVODILA}
\vspace{5mm}
\begin{enumerate}
\item Izpit je sestavljen iz 10 vprašanj.\\
\item Uporaba kalkulatorja ni dovoljena. \\
\item Zapišite le tisto, kar zahteva vprašanje. \\
\item Napačni odgovori ne prinašajo minus točk. \\
\end{enumerate}

% put question in a new page
\newpage
\lhead{\examdate}
\rhead{\academicyear}
\chead{\subject}
\cfoot{\thepage}
\setlength{\footskip}{40pt}

\question{1}
Naštejte nekaj slabosti standardizacije podatkov. \hfill (3~točke)

\question{2}
Zapišite glavne značilnosti inteligentnih sistemov. \hfill (2~točki)

\question{3}
Naštejte vsaj 5 tehnik priprave podatkov in zapišite njihove značilnosti in za kakšen namen se določena tehnika uporablja? \hfill (6~točk)
\begin{itemize}
\item[(i)] 
Zakaj je pomembno izvesti pripravo podatkov preden začnemo učiti klasifikator? \hfill (2~točki)
\item[(ii)] 
Ali je vedno potrebno izvesti postopek priprave podatkov? \hfill (2~točki)
\item[(iii)]Kaj pomeni termin \textbf{podatkovna množica} in kako je sestavljena? \hfill (2~točki)
\end{itemize}

\end{document}